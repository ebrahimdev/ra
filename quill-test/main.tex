\documentclass{article}
\usepackage{amsmath}
\usepackage{graphicx}
\usepackage{cite}

\title{Exploring Curriculum Learning in Neural Networks}
\author{Jane Doe}
\date{\today}

\begin{document}

\maketitle

\begin{abstract}
This survey endeavors to furnish an extensive exploration of alignment methodologies designed for LLMs, in conjunction with the extant capability research in this domain.

Curriculum learning is an approach where models are trained on samples in a meaningful order, typically from easy to hard. This paper explores how curriculum strategies affect convergence speed and generalization in deep learning.
\end{abstract}
 
\section{Introduction}

Deep learning models are typically trained with randomly ordered samples. However, humans and animals learn more effectively when examples are presented in a meaningful sequence. This motivates the idea of curriculum learning.

\section{Related Work}

\cite{bengio2009curriculum} introduced curriculum learning as a formal training paradigm. Several follow-up studies have investigated automatic curriculum generation and self-paced learning.

\section{Methodology}

We propose a curriculum scheduler that ranks training samples based on uncertainty. The curriculum is dynamically updated every epoch based on model predictions.

\section{Experiments}


Such advancements, while garnering significant attention,\nhave concurrently elicited various concerns. The potential of these models is\nundeniably vast; however, they may yield texts that are imprecis

We evaluate on MNIST, CIFAR-10, and a synthetic toy dataset. Metrics include convergence speed, final accuracy, and robustness to noise.

\section{Conclusion}

Curriculum learning improves convergence and generalization across multiple datasets. Future work includes curriculum design for reinforcement learning tasks.

\bibliographystyle{plain}
\bibliography{refs}

\end{document}
