\documentclass{article}
\usepackage{amsmath}
\usepackage{graphicx}
\usepackage{cite}

\title{Exploring Curriculum Learning in Neural Networks}
\author{Jane Doe}
\date{\today}

\begin{document}

\maketitle

\begin{abstract}
This survey endeavors to furnish an extensive exploration of alignment methodologies designed for LLMs, in conjunction with the extant capability research in this domain. but what if.

\end{abstract}
 
\section{Introduction}

let's see if this is working even here. as you can see it does not show if it found a specific citation for this. 


\section{Related Work}

\cite{bengio2009curriculum} introduced curriculum learning as a formal training paradigm. Several follow-up studies have investigated automatic curriculum generation and self-paced learning.

\section{Methodology}

This survey endeavors to furnish an extensive exploration of alignment methodologies designed for LLMs, in conjunction with the extant capability research in this domain.\cite{shen2023large}



\section{Experiments}


We present new evidence of overthinking, where models disregard correct solutions even when explicitly provided, instead continuing to generate unnecessary reasoning steps that often lead to incorrect conclusions.Experiments on three state-of-the-art models using the AIME2024 math benchmark reveal critical limitations in these models ability to integrate corrective information, posing new challenges for achieving robust and interpretable reasoning.\cite{cuesta-ramirez2025large}


\section{Conclusion}





\bibliographystyle{plain}
\bibliography{refs}

\end{document}
